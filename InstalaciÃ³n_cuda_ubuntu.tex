
% Default to the notebook output style

    


% Inherit from the specified cell style.




    
\documentclass{article}

    
    
    \usepackage{graphicx} % Used to insert images
    \usepackage{adjustbox} % Used to constrain images to a maximum size 
    \usepackage{color} % Allow colors to be defined
    \usepackage{enumerate} % Needed for markdown enumerations to work
    \usepackage{geometry} % Used to adjust the document margins
    \usepackage{amsmath} % Equations
    \usepackage{amssymb} % Equations
    \usepackage[mathletters]{ucs} % Extended unicode (utf-8) support
    \usepackage[utf8x]{inputenc} % Allow utf-8 characters in the tex document
    \usepackage{fancyvrb} % verbatim replacement that allows latex
    \usepackage{grffile} % extends the file name processing of package graphics 
                         % to support a larger range 
    % The hyperref package gives us a pdf with properly built
    % internal navigation ('pdf bookmarks' for the table of contents,
    % internal cross-reference links, web links for URLs, etc.)
    \usepackage{hyperref}
    \usepackage{longtable} % longtable support required by pandoc >1.10
    

    
    
    \definecolor{orange}{cmyk}{0,0.4,0.8,0.2}
    \definecolor{darkorange}{rgb}{.71,0.21,0.01}
    \definecolor{darkgreen}{rgb}{.12,.54,.11}
    \definecolor{myteal}{rgb}{.26, .44, .56}
    \definecolor{gray}{gray}{0.45}
    \definecolor{lightgray}{gray}{.95}
    \definecolor{mediumgray}{gray}{.8}
    \definecolor{inputbackground}{rgb}{.95, .95, .85}
    \definecolor{outputbackground}{rgb}{.95, .95, .95}
    \definecolor{traceback}{rgb}{1, .95, .95}
    % ansi colors
    \definecolor{red}{rgb}{.6,0,0}
    \definecolor{green}{rgb}{0,.65,0}
    \definecolor{brown}{rgb}{0.6,0.6,0}
    \definecolor{blue}{rgb}{0,.145,.698}
    \definecolor{purple}{rgb}{.698,.145,.698}
    \definecolor{cyan}{rgb}{0,.698,.698}
    \definecolor{lightgray}{gray}{0.5}
    
    % bright ansi colors
    \definecolor{darkgray}{gray}{0.25}
    \definecolor{lightred}{rgb}{1.0,0.39,0.28}
    \definecolor{lightgreen}{rgb}{0.48,0.99,0.0}
    \definecolor{lightblue}{rgb}{0.53,0.81,0.92}
    \definecolor{lightpurple}{rgb}{0.87,0.63,0.87}
    \definecolor{lightcyan}{rgb}{0.5,1.0,0.83}
    
    % commands and environments needed by pandoc snippets
    % extracted from the output of `pandoc -s`
    \DefineVerbatimEnvironment{Highlighting}{Verbatim}{commandchars=\\\{\}}
    % Add ',fontsize=\small' for more characters per line
    \newenvironment{Shaded}{}{}
    \newcommand{\KeywordTok}[1]{\textcolor[rgb]{0.00,0.44,0.13}{\textbf{{#1}}}}
    \newcommand{\DataTypeTok}[1]{\textcolor[rgb]{0.56,0.13,0.00}{{#1}}}
    \newcommand{\DecValTok}[1]{\textcolor[rgb]{0.25,0.63,0.44}{{#1}}}
    \newcommand{\BaseNTok}[1]{\textcolor[rgb]{0.25,0.63,0.44}{{#1}}}
    \newcommand{\FloatTok}[1]{\textcolor[rgb]{0.25,0.63,0.44}{{#1}}}
    \newcommand{\CharTok}[1]{\textcolor[rgb]{0.25,0.44,0.63}{{#1}}}
    \newcommand{\StringTok}[1]{\textcolor[rgb]{0.25,0.44,0.63}{{#1}}}
    \newcommand{\CommentTok}[1]{\textcolor[rgb]{0.38,0.63,0.69}{\textit{{#1}}}}
    \newcommand{\OtherTok}[1]{\textcolor[rgb]{0.00,0.44,0.13}{{#1}}}
    \newcommand{\AlertTok}[1]{\textcolor[rgb]{1.00,0.00,0.00}{\textbf{{#1}}}}
    \newcommand{\FunctionTok}[1]{\textcolor[rgb]{0.02,0.16,0.49}{{#1}}}
    \newcommand{\RegionMarkerTok}[1]{{#1}}
    \newcommand{\ErrorTok}[1]{\textcolor[rgb]{1.00,0.00,0.00}{\textbf{{#1}}}}
    \newcommand{\NormalTok}[1]{{#1}}
    
    % Define a nice break command that doesn't care if a line doesn't already
    % exist.
    \def\br{\hspace*{\fill} \\* }
    % Math Jax compatability definitions
    \def\gt{>}
    \def\lt{<}
    % Document parameters
    \title{Instalaci?n\_cuda\_ubuntu}
    
    
    

    % Pygments definitions
    
\makeatletter
\def\PY@reset{\let\PY@it=\relax \let\PY@bf=\relax%
    \let\PY@ul=\relax \let\PY@tc=\relax%
    \let\PY@bc=\relax \let\PY@ff=\relax}
\def\PY@tok#1{\csname PY@tok@#1\endcsname}
\def\PY@toks#1+{\ifx\relax#1\empty\else%
    \PY@tok{#1}\expandafter\PY@toks\fi}
\def\PY@do#1{\PY@bc{\PY@tc{\PY@ul{%
    \PY@it{\PY@bf{\PY@ff{#1}}}}}}}
\def\PY#1#2{\PY@reset\PY@toks#1+\relax+\PY@do{#2}}

\expandafter\def\csname PY@tok@gd\endcsname{\def\PY@tc##1{\textcolor[rgb]{0.63,0.00,0.00}{##1}}}
\expandafter\def\csname PY@tok@gu\endcsname{\let\PY@bf=\textbf\def\PY@tc##1{\textcolor[rgb]{0.50,0.00,0.50}{##1}}}
\expandafter\def\csname PY@tok@gt\endcsname{\def\PY@tc##1{\textcolor[rgb]{0.00,0.27,0.87}{##1}}}
\expandafter\def\csname PY@tok@gs\endcsname{\let\PY@bf=\textbf}
\expandafter\def\csname PY@tok@gr\endcsname{\def\PY@tc##1{\textcolor[rgb]{1.00,0.00,0.00}{##1}}}
\expandafter\def\csname PY@tok@cm\endcsname{\let\PY@it=\textit\def\PY@tc##1{\textcolor[rgb]{0.25,0.50,0.50}{##1}}}
\expandafter\def\csname PY@tok@vg\endcsname{\def\PY@tc##1{\textcolor[rgb]{0.10,0.09,0.49}{##1}}}
\expandafter\def\csname PY@tok@m\endcsname{\def\PY@tc##1{\textcolor[rgb]{0.40,0.40,0.40}{##1}}}
\expandafter\def\csname PY@tok@mh\endcsname{\def\PY@tc##1{\textcolor[rgb]{0.40,0.40,0.40}{##1}}}
\expandafter\def\csname PY@tok@go\endcsname{\def\PY@tc##1{\textcolor[rgb]{0.53,0.53,0.53}{##1}}}
\expandafter\def\csname PY@tok@ge\endcsname{\let\PY@it=\textit}
\expandafter\def\csname PY@tok@vc\endcsname{\def\PY@tc##1{\textcolor[rgb]{0.10,0.09,0.49}{##1}}}
\expandafter\def\csname PY@tok@il\endcsname{\def\PY@tc##1{\textcolor[rgb]{0.40,0.40,0.40}{##1}}}
\expandafter\def\csname PY@tok@cs\endcsname{\let\PY@it=\textit\def\PY@tc##1{\textcolor[rgb]{0.25,0.50,0.50}{##1}}}
\expandafter\def\csname PY@tok@cp\endcsname{\def\PY@tc##1{\textcolor[rgb]{0.74,0.48,0.00}{##1}}}
\expandafter\def\csname PY@tok@gi\endcsname{\def\PY@tc##1{\textcolor[rgb]{0.00,0.63,0.00}{##1}}}
\expandafter\def\csname PY@tok@gh\endcsname{\let\PY@bf=\textbf\def\PY@tc##1{\textcolor[rgb]{0.00,0.00,0.50}{##1}}}
\expandafter\def\csname PY@tok@ni\endcsname{\let\PY@bf=\textbf\def\PY@tc##1{\textcolor[rgb]{0.60,0.60,0.60}{##1}}}
\expandafter\def\csname PY@tok@nl\endcsname{\def\PY@tc##1{\textcolor[rgb]{0.63,0.63,0.00}{##1}}}
\expandafter\def\csname PY@tok@nn\endcsname{\let\PY@bf=\textbf\def\PY@tc##1{\textcolor[rgb]{0.00,0.00,1.00}{##1}}}
\expandafter\def\csname PY@tok@no\endcsname{\def\PY@tc##1{\textcolor[rgb]{0.53,0.00,0.00}{##1}}}
\expandafter\def\csname PY@tok@na\endcsname{\def\PY@tc##1{\textcolor[rgb]{0.49,0.56,0.16}{##1}}}
\expandafter\def\csname PY@tok@nb\endcsname{\def\PY@tc##1{\textcolor[rgb]{0.00,0.50,0.00}{##1}}}
\expandafter\def\csname PY@tok@nc\endcsname{\let\PY@bf=\textbf\def\PY@tc##1{\textcolor[rgb]{0.00,0.00,1.00}{##1}}}
\expandafter\def\csname PY@tok@nd\endcsname{\def\PY@tc##1{\textcolor[rgb]{0.67,0.13,1.00}{##1}}}
\expandafter\def\csname PY@tok@ne\endcsname{\let\PY@bf=\textbf\def\PY@tc##1{\textcolor[rgb]{0.82,0.25,0.23}{##1}}}
\expandafter\def\csname PY@tok@nf\endcsname{\def\PY@tc##1{\textcolor[rgb]{0.00,0.00,1.00}{##1}}}
\expandafter\def\csname PY@tok@si\endcsname{\let\PY@bf=\textbf\def\PY@tc##1{\textcolor[rgb]{0.73,0.40,0.53}{##1}}}
\expandafter\def\csname PY@tok@s2\endcsname{\def\PY@tc##1{\textcolor[rgb]{0.73,0.13,0.13}{##1}}}
\expandafter\def\csname PY@tok@vi\endcsname{\def\PY@tc##1{\textcolor[rgb]{0.10,0.09,0.49}{##1}}}
\expandafter\def\csname PY@tok@nt\endcsname{\let\PY@bf=\textbf\def\PY@tc##1{\textcolor[rgb]{0.00,0.50,0.00}{##1}}}
\expandafter\def\csname PY@tok@nv\endcsname{\def\PY@tc##1{\textcolor[rgb]{0.10,0.09,0.49}{##1}}}
\expandafter\def\csname PY@tok@s1\endcsname{\def\PY@tc##1{\textcolor[rgb]{0.73,0.13,0.13}{##1}}}
\expandafter\def\csname PY@tok@sh\endcsname{\def\PY@tc##1{\textcolor[rgb]{0.73,0.13,0.13}{##1}}}
\expandafter\def\csname PY@tok@sc\endcsname{\def\PY@tc##1{\textcolor[rgb]{0.73,0.13,0.13}{##1}}}
\expandafter\def\csname PY@tok@sx\endcsname{\def\PY@tc##1{\textcolor[rgb]{0.00,0.50,0.00}{##1}}}
\expandafter\def\csname PY@tok@bp\endcsname{\def\PY@tc##1{\textcolor[rgb]{0.00,0.50,0.00}{##1}}}
\expandafter\def\csname PY@tok@c1\endcsname{\let\PY@it=\textit\def\PY@tc##1{\textcolor[rgb]{0.25,0.50,0.50}{##1}}}
\expandafter\def\csname PY@tok@kc\endcsname{\let\PY@bf=\textbf\def\PY@tc##1{\textcolor[rgb]{0.00,0.50,0.00}{##1}}}
\expandafter\def\csname PY@tok@c\endcsname{\let\PY@it=\textit\def\PY@tc##1{\textcolor[rgb]{0.25,0.50,0.50}{##1}}}
\expandafter\def\csname PY@tok@mf\endcsname{\def\PY@tc##1{\textcolor[rgb]{0.40,0.40,0.40}{##1}}}
\expandafter\def\csname PY@tok@err\endcsname{\def\PY@bc##1{\setlength{\fboxsep}{0pt}\fcolorbox[rgb]{1.00,0.00,0.00}{1,1,1}{\strut ##1}}}
\expandafter\def\csname PY@tok@kd\endcsname{\let\PY@bf=\textbf\def\PY@tc##1{\textcolor[rgb]{0.00,0.50,0.00}{##1}}}
\expandafter\def\csname PY@tok@ss\endcsname{\def\PY@tc##1{\textcolor[rgb]{0.10,0.09,0.49}{##1}}}
\expandafter\def\csname PY@tok@sr\endcsname{\def\PY@tc##1{\textcolor[rgb]{0.73,0.40,0.53}{##1}}}
\expandafter\def\csname PY@tok@mo\endcsname{\def\PY@tc##1{\textcolor[rgb]{0.40,0.40,0.40}{##1}}}
\expandafter\def\csname PY@tok@kn\endcsname{\let\PY@bf=\textbf\def\PY@tc##1{\textcolor[rgb]{0.00,0.50,0.00}{##1}}}
\expandafter\def\csname PY@tok@mi\endcsname{\def\PY@tc##1{\textcolor[rgb]{0.40,0.40,0.40}{##1}}}
\expandafter\def\csname PY@tok@gp\endcsname{\let\PY@bf=\textbf\def\PY@tc##1{\textcolor[rgb]{0.00,0.00,0.50}{##1}}}
\expandafter\def\csname PY@tok@o\endcsname{\def\PY@tc##1{\textcolor[rgb]{0.40,0.40,0.40}{##1}}}
\expandafter\def\csname PY@tok@kr\endcsname{\let\PY@bf=\textbf\def\PY@tc##1{\textcolor[rgb]{0.00,0.50,0.00}{##1}}}
\expandafter\def\csname PY@tok@s\endcsname{\def\PY@tc##1{\textcolor[rgb]{0.73,0.13,0.13}{##1}}}
\expandafter\def\csname PY@tok@kp\endcsname{\def\PY@tc##1{\textcolor[rgb]{0.00,0.50,0.00}{##1}}}
\expandafter\def\csname PY@tok@w\endcsname{\def\PY@tc##1{\textcolor[rgb]{0.73,0.73,0.73}{##1}}}
\expandafter\def\csname PY@tok@kt\endcsname{\def\PY@tc##1{\textcolor[rgb]{0.69,0.00,0.25}{##1}}}
\expandafter\def\csname PY@tok@ow\endcsname{\let\PY@bf=\textbf\def\PY@tc##1{\textcolor[rgb]{0.67,0.13,1.00}{##1}}}
\expandafter\def\csname PY@tok@sb\endcsname{\def\PY@tc##1{\textcolor[rgb]{0.73,0.13,0.13}{##1}}}
\expandafter\def\csname PY@tok@k\endcsname{\let\PY@bf=\textbf\def\PY@tc##1{\textcolor[rgb]{0.00,0.50,0.00}{##1}}}
\expandafter\def\csname PY@tok@se\endcsname{\let\PY@bf=\textbf\def\PY@tc##1{\textcolor[rgb]{0.73,0.40,0.13}{##1}}}
\expandafter\def\csname PY@tok@sd\endcsname{\let\PY@it=\textit\def\PY@tc##1{\textcolor[rgb]{0.73,0.13,0.13}{##1}}}

\def\PYZbs{\char`\\}
\def\PYZus{\char`\_}
\def\PYZob{\char`\{}
\def\PYZcb{\char`\}}
\def\PYZca{\char`\^}
\def\PYZam{\char`\&}
\def\PYZlt{\char`\<}
\def\PYZgt{\char`\>}
\def\PYZsh{\char`\#}
\def\PYZpc{\char`\%}
\def\PYZdl{\char`\$}
\def\PYZhy{\char`\-}
\def\PYZsq{\char`\'}
\def\PYZdq{\char`\"}
\def\PYZti{\char`\~}
% for compatibility with earlier versions
\def\PYZat{@}
\def\PYZlb{[}
\def\PYZrb{]}
\makeatother


    % Exact colors from NB
    \definecolor{incolor}{rgb}{0.0, 0.0, 0.5}
    \definecolor{outcolor}{rgb}{0.545, 0.0, 0.0}



    
    % Prevent overflowing lines due to hard-to-break entities
    \sloppy 
    % Setup hyperref package
    \hypersetup{
      breaklinks=true,  % so long urls are correctly broken across lines
      colorlinks=true,
      urlcolor=blue,
      linkcolor=darkorange,
      citecolor=darkgreen,
      }
    % Slightly bigger margins than the latex defaults
    
    \geometry{verbose,tmargin=1in,bmargin=1in,lmargin=1in,rmargin=1in}
    
    

    \begin{document}
    
    
    \maketitle
    
    

    

    \section{Instalando CUDA 5.5 y Pycuda en Ubuntu 13.04}


    La instalación de \emph{CUDA} puede resultar un proceso altamente
frustrante (generalmente), sin embargo las fuentes de problemas no son
tantas como uno en principio podría imaginar, así que una vez que se han
detectado éstas se puede instalar el paquete con relativa rapidez.

Es necesario hacer notar que si bien el algoritmo para la instalación de
\emph{CUDA} en \emph{Ubuntu} debería ser muy similar entre las versiones
de ésta distribución, se tienen diferencias muy importantes entre una
versión y otra, i.e.~una puede presentar errores que la versión anterior
no mostraba, o en su defecto puede mostrar menos errores, sin embargo es
algo que nunca pasa. El algoritmo que mostramos aquí es para la
instalación de \emph{CUDA} en \emph{Ubuntu 13.04}. Se puede encontrar
gran cantidad de información en la red acerca de los problemas que se
presentan durante la instalación del paquete en otras versiones, así
como las soluciones que los mismos usuarios han ido encontrando con el
paso del tiempo, si esta guía no sirve completamente para la instalación
en otra versión, le sugerimos al lector buscar ayuda en blogs.


    \subsection{Instalación de controladores de NVIDIA, NVIDIA Toolkit y ejemplos.}


    El proceso de instalación que aquí mostramos está dividido en 3 etapas.
La primera de ellas consiste en instalar los controladores. Para esto es
necesario descargar el archivo \textbf{cuda\_5.5.22\_linux\_64.run} para
\emph{Ubuntu} que se encuentra disponible en la página de \emph{NVIDIA}
https://developer.nvidia.com/cuda-downloads y recordar el lugar en donde
se encuentra guardado el archivo.

Hecho esto podemos proceder con la instalación. El primer paso es
salirse del entorno gráfico, ésto en \emph{Ubuntu} se hace tecleando la
combinación \textbf{Ctrl + Alt + F1}, es importante no entrar en pánico,
puesto que inmediatamente después de teclear esto la computadora
mostrará sólamente la terminal, en lugar del amigable ambiente de
Ubuntu, la terminal pedirá que se acceda a una cuenta, para esto pedirá
primero el nombre de usuario y una vez ingresado éste solicitará la
contraseña; la siguiente observación que hacemos es que mientras se
teclea la contraseña la computadora parece estar ``trabada'', ya que no
muestra asteriscos cada que se teclea un caracter (como es costumbre),
de hecho no muestra nada, pero eso es sólo por seguridad, es necesario
ignorar tal hecho, ingresar la contraseña y teclear \textbf{Enter}.
Ahora que se ha accedido serán necesarios permisos de superusuario,
sugerimos que cada vez que sean necesarios éstos se use el comando
\textbf{sudo} antecediendo al comando en cuestión, sin embargo otra
opción válida es utilizar el comando \textbf{su} y con esto, de ese
punto en adelante todas las órdenes se interpretarán como órdenes dadas
por el superusuario.

El primero de los problemas con el que todo usuario que intenta instalar
\emph{CUDA} se encuentra es que aunque está en una terminal el entorno
gráfico sigue operando en el fondo, y esto interfiere con la instalación
de los controladores de \emph{NVIDIA}, así que es necesario matar por
completo el entorno gráfico, esto se hace ingresando el siguiente
comando en la terminal.

\begin{itemize}
\itemsep1pt\parskip0pt\parsep0pt
\item
  \textbf{sudo service lightdm stop}
\end{itemize}

Esto nos evitará tener (en principio) errores derivados de interferencia
del ambiente gráfico con el instalador. Una vez solucionado este
problema podemos comenzar la instalación, para esto es necesario
movernos a la carpeta en que se ecuentra el archivo .run previamente
descargado y ejecutarlo, sin embargo es necesario cambiar los permisos
del archivo para poderlo ejecutar ya que originalmente no es un archivo
con estas características, existen varios comandos para volver
ejecutable un archivo, uno de ellos es el siguiente:

\begin{itemize}
\itemsep1pt\parskip0pt\parsep0pt
\item
  \textbf{sudo chmod +x cuda\_5.5.22\_linux\_64.run}
\end{itemize}

Ahora contamos con un archivo ejecutable y hasta esta instancia nos es
posible comenzar con la instalación de los controladores de \emph{CUDA},
resaltamos que no instalaremos al mismo tiempo los controladores, el
\emph{toolkit} y los ejemplos, sino que lo haremos por fases instalando
uno a la vez durante tres etapas. Para comenzar con la primera
ejecutamos el archivo .run con el siguiente comando:

\begin{itemize}
\itemsep1pt\parskip0pt\parsep0pt
\item
  \textbf{./cuda\_5.5.22\_linux\_64.run}
\end{itemize}

En ocasiones el instalador detecta conflictos con un codec de
\emph{Ubuntu} (\emph{Noveau}) y arroja un error, si esto sucede, el
mismo instalador genera una archivo auxiliar para no tener este problema
la próxima vez que sea ejecutado; si esto sucede hay que reiniciar el
sistema y nuevamente seguir todos los pasos hasta el momento en que se
ejecuta el archivo .run. Si no se presenta ningún conflicto con
\emph{Noveau} comenzará a correr el instalador sin problemas, al correr
el archivo \emph{.run} se desplegarán los términos y condiciones para el
uso de la paquetería de \emph{NVIDIA}, al terminar preguntará si se
aceptan los términos y condiciones, uno debe ingresar \emph{accept},
posteriormente el instalador informará que está a punto de instalar
software que no cuenta con soporte y preguntará si se desea continuar,
en este caso sólo tecleamos la letra \emph{y} (yes) para continuar;
posteriormente preguntará si se desea instalar el driver, aceptamos de
nuevo con \emph{y}; las siguientes dos opciones que salgan las negaremos
por el momento tecleando \emph{n} en cada una, estas opciones
corresponden a la instalación del \emph{toolkit} y las muestras. Una vez
concluida la instalación del \emph{driver}, que notaremos por que el
instalador muestra una pantalla al final informando si la instalación
fue o no exitosa, podemos volver al ambiente gráfico con el comando:

\begin{itemize}
\itemsep1pt\parskip0pt\parsep0pt
\item
  \textbf{sudo service lightdm start}
\end{itemize}

A partir de este momento podemos continuar la instalación en una
terminal de \emph{Ubuntu}. En la terminal accedemos al directorio en que
se encuentre el archivo, pero antes de ejecutar de nuevo el archivo
\emph{.run} es necesario que instalemos algunos paquetes y desinstalemos
otros.

\emph{NVIDIA} da una lista de prerrequisitos, i.e.~algunos programas y
controladores necesarios para poder llevar a cabo la instalación, sin
embargo debemos proceder cautelosamente en este aspecto, primero
instalamos el siguiente paquete

\begin{itemize}
\itemsep1pt\parskip0pt\parsep0pt
\item
  \textbf{sudo apt-get install freeglut3 freeglut3-dev}
\end{itemize}

Una vez instalado tenemos que remover los compiladores que \emph{Ubuntu
13.04} tiene por default e instalar los ``viejos'' compiladores, esto lo
hacemos con los siguientes comandos:

\begin{itemize}
\itemsep1pt\parskip0pt\parsep0pt
\item
  \textbf{sudo rm /usr/bin/cpp}
\item
  \textbf{sudo rm /usr/bin/gcc}
\item
  \textbf{sudo rm /usr/bin/g++}
\end{itemize}

Ahora instalamos los viejos compiladores y creamos algunas ligas suaves,
con el fin de que cada vez que el sistema mande llamar a los
compiladores con el nombre predefinido en la instalación sepa en dónde
encontrarlos:

\begin{itemize}
\itemsep1pt\parskip0pt\parsep0pt
\item
  \textbf{sudo apt-get install gcc-4.6 g++-4.6}
\item
  \textbf{sudo ln -s /usr/bin/cpp-4.6 /usr/bin/cpp}
\item
  \textbf{sudo ln -s /usr/bin/gcc-4.6 /usr/bin/gcc}
\item
  \textbf{sudo ln -s /usr/bin/g++-4.6 /usr/bin/g++}
\end{itemize}

Después instalamos los prerrequisitos faltantes con

\begin{itemize}
\itemsep1pt\parskip0pt\parsep0pt
\item
  \textbf{sudo apt-get install freeglut3-dev build-essential libx11-dev
  libxmu-dev libxi-dev libgl1-mesa-glx libglu1-mesa libglu1-mesa-dev}
\end{itemize}

Si se concluyeron satisfactoriamente todos los pasos anteriores nos
encontramos en condiciones para instalar tanto el \emph{toolkit} como
los ejemplos, para este fin volvemos a ejecutar el archivo .run pero en
esta etapa sólo elegimos la opción que nos permite instalar el
\emph{toolkit} y decimos que no a las otras dos opciones; al aceptar la
opción del \emph{toolkit} el instalador nos preguntará la carpeta en la
que deseamos instalarlo y nos da una dirección predefinida, sugerimos
que se acepte esta direción, sólo hay que pulsar la tecla \emph{Enter} y
con esto el instalador podnrá las cosas en la carpeta antes mencionada.
Concluido este paso es necesario agregar algunas lineas al archivo
\emph{bash.bashrc} que se encuentra localizado en la carpeta
\emph{/etc/}

\begin{itemize}
\itemsep1pt\parskip0pt\parsep0pt
\item
  export PATH=/usr/local/cuda-5.5/bin:\$PATH
\item
  export
  LD\_LIBRARY\_PATH=/usr/local/cuda-5.5/lib:/usr/local/cuda-5.5/lib64:\$LD\_LIBRARY\_PATH
\item
  export PATH=/usr/local/cuda/bin:\$PATH
\end{itemize}

Por último instalaremos los ejemplos, estrictamente hablando, no
necesitamos instalarlos para que funcione \emph{CUDA}, sin embargo son
útiles para verificar si la instalación se llevó a cabo correctamente en
el sistema. Para instalar los ejemplos volvemos a ejecutar a nuestro
viejo amigo el archivo \emph{.run} y esta vez sólo damos como válida la
opción de instalar los ejemplos, al aceptar esta opción el instalador
preguntará en qué carpeta se encuentra el \emph{toolkit} y dónde
deseamos instalar los ejemplos, si durante la instalación del
\emph{toolkit} elegimos que se instalara en la carpeta predefinida
entonces sólo tecleamos \emph{Enter} y dejamos que el instalador busque
en las carpetas predefinidas, de otra manera es necesario ingrsar las
ruta de la carpeta en donde se instaló el \emph{toolkit} y donde se
instalarán los ejemplos.

Si se ejecuta el paso anterior sin problema hemos terminado, sin embargo
puede surgir un problema al momento de instalar los ejemplos, en
terminal se muestra el mensaje de que la instalación de los ejemplos ha
fallado, si esto pasa, no debe entrar en pánico, si la cuenta que usa en
terminal no es de superusuario tecleé

\begin{itemize}
\itemsep1pt\parskip0pt\parsep0pt
\item
  \textbf{cd}
\end{itemize}

Si está logeado como superusuario muévase a la carpeta home del usuario
dueño de la cuenta en donde se instalaron los drivers

\begin{itemize}
\itemsep1pt\parskip0pt\parsep0pt
\item
  \textbf{cd /home/usuario/}
\end{itemize}

Dentro de esta carpeta teclee el siguiente comando

\begin{itemize}
\itemsep1pt\parskip0pt\parsep0pt
\item
  \textbf{cd NVIDIA\_CUDA-5.5\_Samples/NVIDIA\_CUDA-5.5\_Samples}
\end{itemize}

y con el fin de compilar los ejemplos ejecute el siguiente comando

\begin{itemize}
\itemsep1pt\parskip0pt\parsep0pt
\item
  \textbf{make}
\end{itemize}

Éste paso debería ejecutarse sin problemas, y una vez finalizado el
proceso (que puede durar del orden de 30 minutos) tendremos los ejemplos
instalados correctamente. Al terminar este paso es necesario que
reinicie el equipo.

Lo siguiente que tenemos que hacer es comprobar si \emph{CUDA} funciona
correctamente, para esto sugerimos lo siguiente; muévase a la siguiente
carpeta, con el fin de obtener una grata impresión (y vaya impresión)

\begin{itemize}
\itemsep1pt\parskip0pt\parsep0pt
\item
  \textbf{cd
  NVIDIA\_CUDA-5.5\_Samples/NVIDIA\_CUDA-5.5\_Samples/5\_simulations/fluidsGL}
\end{itemize}

Aquí es donde la diversión comienza, prepare una bolsa de palomitas de
maíz (coma sólo con una mano porque ocupará la otra), y consiga un
monitor enorme y de buena resolución (no es realmente necesario), ya que
tenga todo esto listo tecleé

\begin{itemize}
\itemsep1pt\parskip0pt\parsep0pt
\item
  \textbf{./fluidsGL}
\end{itemize}

Aparecerá una ventana con un fondo por así decirlo granulado, tome el
ratón con la mano que no tenga mantequilla de palomitas mantenga
presionado cualquiera de los tres botones del ratón y muévalo en su
dirección predilecta.

\emph{Bienvenido a CUDA}


    \subsection{Instalación de PyCUDA}


    Para la instalación de PyCUDA y en general para el cálculo científico
utilizando Python necesitamos la librería NumPy. NumPy es el paquete
fundamental para cómputo científico con Python, contiene entre otras
cosas la capacidad de usar arreglos en N dimensiones, funciones
sofisticadas, herramientas para integrar códigos escritos en C/C++ y
Fortran a Python y capacidad de realizar operaciones de álgebra lineal,
transformadas de Fourier y números aleatorios. La instalación de NumPy
es muy sencilla, para esto sólo es necesario ejecutar el siguiente
comando en una terminal

\begin{itemize}
\itemsep1pt\parskip0pt\parsep0pt
\item
  \textbf{sudo apt-get install python-numpy -y}
\end{itemize}

Con NumPy en la computadora tenemos al alcance un sinnúmero de
facilidades que de otra forma resultaría un poco engorroso de programar,
el punto de NumPy es evitar perder el tiempo programando por ejemplo la
función seno para después utilizarla, entre otras ventajas qué reporta
usar éste paquete. Ya que contemos con NumPy en la computadora será
necesario agregar otras librerías que se utilizan durante el proceso de
instalación de PyCUDA, lo cual hacemos con el siguiente comando

\begin{itemize}
\itemsep1pt\parskip0pt\parsep0pt
\item
  \textbf{sudo apt-get install build-essential python-dev
  python-setuptools libboost-python-dev libboost-thread-dev -y}
\end{itemize}

Después de instalar estos paquetes estamos en condiciones de iniciar la
siguiente etapa en la instalación de PyCUDA. Una vez contando con los
prerrequisitos es necesario descargar el paquete de PyCUDA que se
encuentra en la siguiente dirección https://pypi.python.org/pypi/pycuda.
Ahora tenemos el paquete en nuestra computadora. Posteriormente nos
movemos con la consola hasta la carpeta donde guardamos el archivo
comprimido y lo desempaquetamos tecleando esto en la terminal

\begin{itemize}
\itemsep1pt\parskip0pt\parsep0pt
\item
  \textbf{tar xzvf pycuda-VERSION.tar.gz}
\end{itemize}

En donde en lugar de \emph{VERSION} ponemos los números correspondientes
a la versión de PyCUDA que descargamos. Esto a su vez crea una carpeta
con todos los archivos que el sistema utilizará para instalar PyCUDA, es
necesario que de aquí en adelante los comandos se ejecuten en la carpeta
que se creó al comprimir el archivo, por lo tanto debemos movernos hacia
ella, lo cual logramos metiendo la siguiente instrucción en la terminal

\begin{itemize}
\itemsep1pt\parskip0pt\parsep0pt
\item
  \textbf{cd pycuda-VERSION}
\end{itemize}

Una vez en la carpeta tenemos que seguir tres pasos: \emph{configurar},
que básicamente es preparar al sistema para llevar a cabo la
instalación; \emph{preparar}, que consiste en alistar el programa para
ser instalado, y por último \emph{instalar}. Para configurar insertamos
cuidadosamente el siguiente comando en la terminal, haciendo unas
pequeñas consideraciones; el la siguiente instrucción
--python-exe=/usr/bin/python\textbf{VER} cambiamos \textbf{VER} por la
versión de \emph{Python} que estemos utilizando y hacemos esto mismo
para la instrucción
--boost-python-libname=boost\_python-mt-py\textbf{VER}

\begin{itemize}
\itemsep1pt\parskip0pt\parsep0pt
\item
  \textbf{./configure.py --python-exe=/usr/bin/pythonVER
  --cuda-root=/usr/local/cuda --cudadrv-lib-dir=/usr/lib
  --boost-inc-dir=/usr/include --boost-lib-dir=/usr/lib
  --boost-python-libname=boost\_python-mt-pyVER
  --boost-thread-libname=boost\_thread-mt --no-use-shipped-boost}
\end{itemize}

Posteriormente debemos hacer un pequeño cambio en el archivo
\emph{siteconfig.py} que se encuentra en esta misma carpeta remplazamos
el valor asociado a \textbf{CUDA\_ENABLE\_GL} cambiando la linea de
código respectiva por lo siguiente

\begin{itemize}
\itemsep1pt\parskip0pt\parsep0pt
\item
  \emph{CUDA\_ENABLE\_GL = True}
\end{itemize}

Ahora que tenemos el sistema configurado lo siguiente que debemos hacer
es avisarle al programa que se prepare para ser instalado, la forma de
avisarle es por medio del siguiente comando

\begin{itemize}
\itemsep1pt\parskip0pt\parsep0pt
\item
  \textbf{make -j 4}
\end{itemize}

Ya le avisamos al sistema que vamos a instalar, y el programa ya está
preparado para salir a la fiesta, lo siguiente que debemos hacer es dar
inicio a la fiesta, es decir, instalar

\begin{itemize}
\itemsep1pt\parskip0pt\parsep0pt
\item
  \textbf{sudo python setup.py install}
\end{itemize}

Y listo, a diferencia de CUDA, PyCUDA es muy amable a la hora de ser
instalado, es decir, es muy raro que se presenten errores durante la
instalación, o por decirlo de otra forma los errores más frecuentes que
se presentan a la hora de instalar PyCUDA son los errores de dedo al
poner el comando en la terminal.

Con CUDA y PyCUDA instalado está usted en condiciones de utilizar y
practicar lo que aprenda a lo largo de estas notas, programar es una
tarea que involucra mucha paciencia y la mejor forma de aprender a
hacerlo es practicando, animamos al lector a agregar detalles a los
programas que aquí mostremos y dejar volar su imaginación con los
problemas que presentemos, programar es una tarea altamente creativa. Es
de esperarse que haya errores en los códigos, nunca salen bien a la
primera, sin embargo como el lector sabrá, o se dará cuenta, la mayor
parte del tiempo dedicado a la programación se gasta revisando en qué
lugares tiene errores el código, así que si pasa más tiempo revisando y
corrigiendo errores que programando deje que su espíritu descanse al
saber que todos sufrimos el mismo martirio.

\emph{Bienvenido a PyCUDA}


    % Add a bibliography block to the postdoc
    
    
    
    \end{document}
